% german language
\usepackage{polyglossia}
\setdefaultlanguage{english}

% intelligent quotation marks, language and nesting sensitive
\usepackage[autostyle]{csquotes}

% microtypographical features, makes the text look nicer on the small scale
\usepackage{microtype}

%------------------------------------------------------------------------------
%---------------------------- Numbers and Units -------------------------------
%------------------------------------------------------------------------------

\usepackage[
  locale=US,
  separate-uncertainty=true,
  group-separator={\,},
  group-minimum-digits={3},
  group-digits=true,
  per-mode=symbol-or-fraction,
  tophrase={{~to~}},
  detect-all,
  binary-units=true
]{siunitx}
\sisetup{
  math-micro=\text{µ},
  text-micro=µ,
  quotient-mode=fraction,
  fraction-function=\sfrac
  }
\DeclareSIUnit\year{a}
\DeclareSIUnit\eqminute{min}
\DeclareSIUnit\minute{minute}
\DeclareSIUnit\minutes{minutes}
\DeclareSIUnit\hours{hours}
\DeclareSIUnit\electron{e}
\DeclareSIUnit\pixel{pixel}
\DeclareSIUnit\pixels{pixels}

%------------------------------------------------------------------------------
%------------------------ Für die Matheumgebung--------------------------------
%------------------------------------------------------------------------------

\usepackage{amsmath}
\usepackage{amssymb}
\usepackage{mathtools}

% Enable Unicode-Math and follow the ISO-Standards for typesetting math
\usepackage[
  math-style=ISO,
  bold-style=ISO,
  sans-style=italic,
  nabla=upright,
  partial=upright,
]{unicode-math}
\setmathfont{Latin Modern Math}

% nice, small fracs for the text with \sfrac{}{}
\usepackage{xfrac}  

%------------------------------------------------------------------------------
%-------------------------------- graphics -------------------------------------
%------------------------------------------------------------------------------

\usepackage{graphicx}
\usepackage{grffile}

% Multiple figures next to each other
\usepackage[labelformat=simple]{subcaption}
\renewcommand{\thesubfigure}{(\alph{subfigure})}

% allow figures to be placed in the running text by default:
\usepackage{scrhack}
\usepackage{float}
\floatplacement{figure}{htbp}
\floatplacement{table}{htbp}

% keep figures and tables in the section
\usepackage[section, below]{placeins}

%------------------------------------------------------------------------------
%-------------------------------- tables  -------------------------------------
%------------------------------------------------------------------------------

\usepackage{multirow}
\usepackage{booktabs}       % stellt \toprule, \midrule, \bottomrule


%------------------------------------------------------------------------------
%------------------------------ Bibliographie ---------------------------------
%------------------------------------------------------------------------------

\usepackage[
  backend=biber,   % use modern biber backend
  autolang=hyphen, % load hyphenation rules for if language of bibentry is not
                   % german, has to be loaded with \setotherlanguages
                   % in the references.bib use langid={en} for english sources
  style=authoryear, % Citation style shows author and year
  uniquelist=minyear, % Only add other authors, when the year is the same
  maxcitenames=1, % only show the first other inside of the text
  maxbibnames=99,
  uniquelist=false,
  uniquename=false
]{biblatex}
\addbibresource{Tex_global/lit.bib}  % die Bibliographie einbinden
\DefineBibliographyStrings{german}{andothers = {{et\,al\adddot}}}
\setlength\bibitemsep{1.5\itemsep} % Bigger distance between two bib entrys


%------------------------------------------------------------------------------
%---------------------- customize list environments ---------------------------
%------------------------------------------------------------------------------

\usepackage{enumitem}


%------------------------------------------------------------------------------
%------------------------------ Sonstiges: ------------------------------------
%------------------------------------------------------------------------------



% hyperlinks
\usepackage[pdfusetitle,unicode,ocgcolorlinks]{hyperref}

\hypersetup{
    colorlinks,
    linkcolor={red!50!black},
    citecolor={blue!50!black},
    urlcolor={blue!80!black}
}

% Craaazt ocgcolorlinksstuff
\makeatletter
\AtBeginDocument{%
    \newlength{\temp@x}%
    \newlength{\temp@y}%
    \newlength{\temp@w}%
    \newlength{\temp@h}%
    \def\my@coords#1#2#3#4{%
      \setlength{\temp@x}{#1}%
      \setlength{\temp@y}{#2}%
      \setlength{\temp@w}{#3}%
      \setlength{\temp@h}{#4}%
      \adjustlengths{}%
      \my@pdfliteral{\strip@pt\temp@x\space\strip@pt\temp@y\space\strip@pt\temp@w\space\strip@pt\temp@h\space re}}%
    \ifpdf
      \typeout{In PDF mode}%
      \def\my@pdfliteral#1{\pdfliteral page{#1}}% I don't know why % this command...
      \def\adjustlengths{}%
    \fi
    \ifxetex
      \def\my@pdfliteral #1{\special{pdf: literal direct #1}}% isn't equivalent to this one
      \def\adjustlengths{\setlength{\temp@h}{-\temp@h}\addtolength{\temp@y}{1in}\addtolength{\temp@x}{-1in}}%
    \fi%
    \def\Hy@colorlink#1{%
      \begingroup
        \ifHy@ocgcolorlinks
          \def\Hy@ocgcolor{#1}%
          \my@pdfliteral{q}%
          \my@pdfliteral{7 Tr}% Set text mode to clipping-only
        \else
          \HyColor@UseColor#1%
        \fi
    }%
    \def\Hy@endcolorlink{%
      \ifHy@ocgcolorlinks%
        \my@pdfliteral{/OC/OCPrint BDC}%
        \my@coords{0pt}{0pt}{\pdfpagewidth}{\pdfpageheight}%
        \my@pdfliteral{F}% Fill clipping path (the url's text) with
                           % current color
        %
        \my@pdfliteral{EMC/OC/OCView BDC}%
        \begingroup%
          \expandafter\HyColor@UseColor\Hy@ocgcolor%
          \my@coords{0pt}{0pt}{\pdfpagewidth}{\pdfpageheight}%
          \my@pdfliteral{F}% Fill clipping path (the url's text)
                             % with \Hy@ocgcolor
        \endgroup%
        \my@pdfliteral{EMC}%
        \my@pdfliteral{0 Tr}% Reset text to normal mode
        \my@pdfliteral{Q}%
      \fi
      \endgroup
    }%
}
\makeatother


\usepackage{bookmark}
\usepackage[shortcuts]{extdash}

% acronyms
\usepackage[printonlyused]{acronym}

% Distance etween lines of 1.5
\usepackage{setspace}
\singlespacing

% No widow or orphan lines
\usepackage[defaultlines=5, all]{nowidow}

%%%%%%%%%%%%%%%%%%%%%%%%%%%%%%%%%%%
%
%
% Own commands
%
%
%%%%%%%%%%%%%%%%%%%%%%%%%%%%%%%%%%%

\newcommand{\tipname}{TIP}
\newcommand{\notename}{NOTE}
\newcommand{\importantname}{IMPORTANT}
\newcommand{\sphire}{\ac{SPHIRE}}
\newcommand{\unblur}{\textprogram{Unblur}}
\newcommand{\summovie}{\textprogram{Summovie}}

\newcommand{\textcommand}[1]{\textbf{\textit{#1}}}
\newcommand{\textbutton}[1]{{\textbf{\textit{#1}}}}
\newcommand{\textfolder}[1]{\textbf{#1}}
\newcommand{\textfile}[1]{\textbf{#1}}
\newcommand{\textprogram}[1]{\textbf{#1}}
\newcommand{\textlabel}[1]{\textbf{#1}}
\newcommand{\textkeyboard}[1]{\textbf{#1}}

\newcommand{\rednum}[1]{{\color{red}#1}}
\newcommand{\purplenum}[1]{{\color{purple}#1}}

\newlength{\parawidth}%
\newlength{\valuewidth}%
\setlength{\parawidth}{5cm}%
\setlength{\valuewidth}{4cm}%

\newlength{\guiwidth}
\setlength{\guiwidth}{\linewidth}

%%%%%%%%%%%%%%%%%%%%%%%%%%%%%%%%%%%
%
%
% Indent environment
%
%
%%%%%%%%%%%%%%%%%%%%%%%%%%%%%%%%%%%

% fot the adjustwidth environments
\usepackage{changepage}

%%%%%%%%%%%%%%%%%%%%%%%%%%%%%%%%%%%
%
%
% Terminal command environment
%
%
%%%%%%%%%%%%%%%%%%%%%%%%%%%%%%%%%%%

\newenvironment{terminal}
{\terminalfont\begin{adjustwidth}{1em}{}\scriptsize}
{\end{adjustwidth}}

\newenvironment{terminalitem}
{\terminalfont\begin{adjustwidth}{0em}{}\scriptsize}
{\end{adjustwidth}}

%%%%%%%%%%%%%%%%%%%%%%%%%%%%%%%%%%%
%
%
% Tip command environment
%
%
%%%%%%%%%%%%%%%%%%%%%%%%%%%%%%%%%%%

\newenvironment{tip}
{\tipfont\color{tip}\begin{adjustwidth}{1em}{}\textbf{\tipname:}\itshape}
{\end{adjustwidth}}

\newenvironment{tipitem}
{\tipfont\color{tip}\begin{adjustwidth}{0em}{}\textbf{\tipname:}\itshape}
{\end{adjustwidth}}

%%%%%%%%%%%%%%%%%%%%%%%%%%%%%%%%%%%
%
%
% Note command environment
%
%
%%%%%%%%%%%%%%%%%%%%%%%%%%%%%%%%%%%

\newenvironment{note}
{\notefont\color{note}\begin{adjustwidth}{1em}{}\textbf{\notename:}\itshape}
{\end{adjustwidth}}

\newenvironment{noteitem}
{\notefont\color{note}\begin{adjustwidth}{0em}{}\textbf{\notename:}\itshape}
{\end{adjustwidth}}

\newcommand{\notetext}[1]{{\notefont\color{note}\textbf{\notename:} #1}}


%%%%%%%%%%%%%%%%%%%%%%%%%%%%%%%%%%%
%
%
% Stdout environment
%
%
%%%%%%%%%%%%%%%%%%%%%%%%%%%%%%%%%%%

\newenvironment{stdout}
{\stdfont\begin{adjustwidth}{1em}{}\scriptsize}
{\end{adjustwidth}}

\newenvironment{stdoutitem}
{\stdfont\begin{adjustwidth}{0em}{}\scriptsize}
{\end{adjustwidth}}

%%%%%%%%%%%%%%%%%%%%%%%%%%%%%%%%%%%
%
%
% Important environment
%
%
%%%%%%%%%%%%%%%%%%%%%%%%%%%%%%%%%%%

\newenvironment{important}
{\importantfont\color{important}\begin{adjustwidth}{1em}{}\textbf{\importantname:}}
{\end{adjustwidth}}

\newenvironment{importantitem}
{\importantfont\color{important}\begin{adjustwidth}{0em}{}\textbf{\importantname:}}
{\end{adjustwidth}}

%%%%%%%%%%%%%%%%%%%%%%%%%%%%%%%%%%%
%
%
% Advanced environment
%
%
%%%%%%%%%%%%%%%%%%%%%%%%%%%%%%%%%%%

\usepackage{fancybox}

\newenvironment{advanced}[1]
{\advancedfont\scriptsize\begin{adjustwidth}{0.025\linewidth}{}\begin{Sbox}\begin{minipage}{0.96\linewidth}\color{advanced_head}\textbf{#1}\color{advanced}\medskip\\}
{\end{minipage}\end{Sbox}\fbox{\TheSbox}\end{adjustwidth}}

\newenvironment{issue}[1]
{\issuefont\scriptsize\begin{adjustwidth}{0.025\linewidth}{}\begin{Sbox}\begin{minipage}{0.96\linewidth}\color{issue_head}\textbf{#1}\color{issue}\medskip\\}
{\end{minipage}\end{Sbox}\fbox{\TheSbox}\end{adjustwidth}}

%%%%%%%%%%%%%%%%%%%%%%%%%%%%%%%%%%%
%
%
% Own itemize
%
%
%%%%%%%%%%%%%%%%%%%%%%%%%%%%%%%%%%%

\newenvironment{myitemize}
{\begin{itemize}[leftmargin=!,labelindent=0em,itemindent=-1em]}
{\end{itemize}}


%%%%%%%%%%%%%%%%%%%%%%%%%%%%%%%%%%%
%
%
% Hyperref links
%
%
%%%%%%%%%%%%%%%%%%%%%%%%%%%%%%%%%%%

\newcommand{\wikiunblur}{http://sphire.mpg.de/wiki/doku.php?id=pipeline:movie:sxunblur}
\newcommand{\wikiunblurassess}{http://sphire.mpg.de/wiki/doku.php?id=pipeline:movie:sxgui_unblur}
\newcommand{\wikicter}{http://sphire.mpg.de/wiki/doku.php?id=pipeline:cter:sxcter}
\newcommand{\wikicterassess}{http://sphire.mpg.de/wiki/doku.php?id=pipeline:cter:sxgui_cter}
\newcommand{\wikireliontosparx}{http://sphire.mpg.de/wiki/doku.php?id=howto:relion2sparx}
\newcommand{\wikiisac}{http://sphire.mpg.de/wiki/doku.php?id=pipeline:isac:sxisac}
\newcommand{\wikiviper}{http://sphire.mpg.de/wiki/doku.php?id=pipeline:viper:start}
\newcommand{\wikimeridien}{http://sphire.mpg.de/wiki/doku.php?id=pipeline:meridien:sxmeridien}
\newcommand{\wikisharpening}{http://sphire.mpg.de/wiki/doku.php?id=pipeline:utilities:sxprocess}
\newcommand{\wikisort}{http://sphire.mpg.de/wiki/doku.php?id=pipeline:sort3d:start}
\newcommand{\wikilocalres}{http://sphire.mpg.de/wiki/doku.php?id=howto:submissions}
\newcommand{\wikiproject}{http://sphire.mpg.de/wiki/doku.php?id=pipeline:project:start}
\newcommand{\wikimask}{http://sphire.mpg.de/wiki/doku.php?id=pipeline:utilities:sxprocess}
\newcommand{\wikiangle}{http://sphire.mpg.de/wiki/doku.php?id=pipeline:utilities:sxprocess}
\newcommand{\wikitemplate}{http://sphire.mpg.de/wiki/doku.php?id=howto:submissions}
\newcommand{\wikibeautify}{http://sphire.mpg.de/wiki/doku.php?id=pipeline:isac:start}

\newcommand{\wikigrigorieff}{http://grigoriefflab.janelia.org/}

\newcommand{\wikieman}{http://blake.bcm.edu/emanwiki/EMAN2/Programs/e2boxer}

\newcommand{\wikidownload}{http://sphire.mpg.de/wiki/doku.php?id=howto:download}
\newcommand{\webdownload}{http://sphire.mpg.de/downloads/sphire_beta_20161216.tar.gz}

\newcommand{\demodownloadname}{sphire\_testdata\_tcda1\_20161216\_movies.tar.gz}
\newcommand{\demodownload}{http://sphire.mpg.de/downloads/sphire_testdata_tcda1_20161216_movies.tar.gz}
